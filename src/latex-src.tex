\documentclass[a4paper,10pt]{letter}
\usepackage[utf8]{inputenc}
\usepackage[spanish]{babel}

\begin{document}
Si simplemente se decea escribir texto en laTex,
sin complicadas f\'ormulas matem\'aticas o efectos especiales
como cambios de fuente, entonces simplemente tiene que escribir
en español normalmente.$\\par$
Si decea cambiar de párrado ha de dejar una línea en blanco o bien
utilizar el comando $\\par$.
No es necesario preocuparse de la sangría de los párrados:
todos los párrafos se sangrarán automáticamente con la excepción
del primer párrafo de una sección.\\
Se ha de distinguir entre la comilla simple 'izquierda'
y la comilla simple `derecha' cuando se escribe en el ordenador
En el caso de que se quieran utilizar comillas dobles se han de
escribir dos caracteres `comilla simple´ seguidos, esto es
''comillas dobles''.\\
También se ha de tener cuidado con los guiones: se utiliza un único
guion para la separación de sílabas, mientras se utilizan
tres guiones seguidos para producir un guion de los que se usan 
como signo de puntuación --- como en esta oración. 
\end{document}
 
